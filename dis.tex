
\documentclass[11pt]{ncsuthesis4}

\usepackage{amsfonts}
\usepackage{amssymb}
\usepackage{fancyvrb}
\usepackage{epsfig}
\usepackage{graphicx} 
\usepackage{latexsym}
% ,graphicx,lscape,dectab,harvard} %latex2e
% \input \path psfig.sty %an excellent utility for including figures-Terry
\pagestyle{plain} %{empty} to get Appendix without page number


\newcommand{\nullset}[0]{\varnothing}
% re-enable italics -- they ARE allowed!
% \renewcommand{\em}[0]{}   % disable italics because they're not allowed!
% \newcommand{\ctt}[0]{}
\newcommand{\ctt}[0]{\tt}

\begin{document}

\newcommand{\puttitle}[0]{Abstraction-Based Generation of Finite
      State Models \\from C Programs}

\title{\puttitle}

\author{Daniel C. DuVarney}
\degreeyear{2002}
\degree{Doctor of Philosophy}
\chair{Dr. S. Purushothaman Iyer}
\memberII{Dr. W. Rance Cleaveland II}
\memberIII{Dr. K. C. Tai}
\memberIV{Dr. John W. Baugh, Jr.}
\numberofmembers{4}

\newpage
\thispagestyle{empty}
\begin{center}
\begin{large}
\textbf{ABSTRACT}
\end{large}
\end{center}
DUVARNEY, DANIEL C. \puttitle. (Under the direction of Associate Professor S. Purushothaman Iyer).

\renewcommand{\baselinestretch}{2}

\vspace{0.2in}

Model checking is a major advancement in the quest for
practical automatic verification methods for computer systems,
and
has been effectively used to discover
flaws in real-world hardware systems.
Unfortunately, applying model-checking techniques to
software systems
has proved to be more difficult, due to the large number of
states and irregular transitions of such systems.
One promising method for generating reasonably-sized
models from programs is the use of {\em data abstraction},
in which the program data is mapped from a large set of possible
values to a much smaller set of abstract values.
This thesis develops a method which,
given a program in the C language and an abstraction mapping,
allows the automatic construction of an
abstract labeled transition system (LTS), which
is much smaller than the concrete LTS
(the LTS which would be generated without the benefit of abstraction).
The method is shown to be {\em sound} in the sense that
if a program is well-behaved in its use of
pointers, then any linear temporal logic formula
which holds true for the corresponding abstract LTS 
will also hold true for the concrete LTS.
Furthermore, if a design exists in the form of a transition system,
then the abstract LTS can be checked against the design for bisimilarity.
Bisimilarity ensures that the program is a faithful implementation of
the design.
A suite of software tools has been implemented based upon
the theory. These tools interface
with the Concurrency Workbench, a model checking system.
A case study is presented which shows the practicality of this technique
for verifying real-world C programs.

\field{Department of Computer Science}
\campus{Raleigh}

\maketitle

\begin{frontmatter}

\begin{dedication}
\begin{center}
\null\vfil
Your dedication goes here $\ldots$
\vfil\null
\end{center}
\end{dedication}

\begin{center}
\begin{large}
\textbf{Biography}
\end{large}
\end{center}
Put your bio here ...

\begin{acknowledgements}
Acknowlegements go here...

\end{acknowledgements}

\tableofcontents
\listoffigures
\listoftables

\end{frontmatter}

\chapter{Introduction}

Summary and overview of thesis

\chapter{The Problem}

A statement of the problem

\chapter{Related Work}

Previous attempts to solve the problem

\chapter{Theory}

A theoretical solution

\chapter{Implementation}

An implementation based on the theory

\chapter{Case study}

Experimental Results~\cite{mlton01}

\chapter{Conclusion}

Why your work is better and future work~\cite{am91}



% \nocite{*}
\bibliographystyle{plain}
\bibliography{dis}

\end{document}
